%%%%%%%%%%%%%%%%%%%%%%%%%%%%%%%%%%%%%%%%%%%%%%%%%%%%%%%%%%%%%%%%%%%%%%%
% QE-GIPAW user's manual
%%%%%%%%%%%%%%%%%%%%%%%%%%%%%%%%%%%%%%%%%%%%%%%%%%%%%%%%%%%%%%%%%%%%%%%
\documentclass[a4paper,11pt,twoside]{article}
\usepackage[utf8]{inputenc}             % For accents and weird symbols
%\usepackage{braket}                     % bra-ket's
\usepackage{amsmath}                    % extra mathematical symbols
\usepackage{bm}                         % bold mathematical symbols
\usepackage{verbatim}                   % insert verbatim files
\usepackage[hang,small,bf]{caption}     % customizable captions
\usepackage{psfrag}                     % smart postscript substitutions
\usepackage{fancyvrb}                   % for including source code
\usepackage[in]{fullpage}               % use all page
\usepackage[pdftex]{graphicx}           % to insert figures
\usepackage{epstopdf}                   % convert eps to pdf
\DeclareGraphicsExtensions{.pdf,.png,.jpg,.eps}
\usepackage{hyperref}                   % crossref in PDF
\hypersetup{pdftex, colorlinks=true, linkcolor=blue, citecolor=blue,
            filecolor=blue, urlcolor=blue, pdftitle=, pdfauthor=Davide Ceresoli,
            pdfsubject=QE-GIPAW user's manual, pdfkeywords=}

%===================================================================================================
\title{QE-GIPAW user's manual}
\author{Davide Ceresoli, \texttt{dceresoli@gmail.com}}
\date{Last revision: \today}
%===================================================================================================

%===================================================================================================
\begin{document}
%===================================================================================================
\maketitle
\thispagestyle{empty}
%\tableofcontents
%\clearpage

%===================================================================================================
\section{Introduction}
%===================================================================================================
QE-GIPAW is an improved version of the GIPAW once code distributed in
Quantum-Espresso. Starting from QE 4.3, QE-GIPAW is distributed as a
stand-alone package that to be compiled against Quantum-Espresso.

%===================================================================================================
\section{Features}
%===================================================================================================
\begin{itemize}
  \item Periodic and isolated systems
  \item Norm-conserving and ultrasoft pseudopotentials
  \item Parallelization over k-points (pools) and g-vectors
  \item Automatic checkpoint and restart
  \item Magnetic susceptibility
  \item NMR chemical shielding tensors~\cite{pickard01,yates07}
  \item Electric Field Gradients (EFG)
  \item EPR g-tensor~\cite{pickard02}
  \item Hyperfine couplings
\end{itemize}

%===================================================================================================
\section{Author contributions}
%===================================================================================================
\begin{itemize}
  \item D. Ceresoli: bare susceptibility, hyperfine core polarization
  \item A. P. Seitsonen and U. Gerstmann: GIPAW reconstruction 
  \item E. Kuçukbenli: ultrasoft and PAW pseudopotentials 
  \item S. de Gironcoli: restart and bug fixes 
\end{itemize}
For further information and to report bugs, please contact:

%===================================================================================================
\section{Quick build instructions}
%===================================================================================================
QE-GIPAW can be configured and compiled automatically once you download Quantum-Espresso.
\begin{enumerate}
\item Configure Quantum-Espresso (QE). If you have problems, read the
QE user's guide at \url{http://www.quantum-espresso.org/user_guide/user_guide.html}
\item Type:
\begin{verbatim}
    make gipaw
\end{verbatim}
This will download from QE-forge the latest stable version of QE-GIPAW
and it will compile it.
\end{enumerate}

%===================================================================================================
\section{Build instructions}
%===================================================================================================
If you don't have a direct Internet connection on your machine, or if you want
to build a different version of the code, or even the SVN version:
\begin{enumerate}
\item Configure and compile Quantum-Espresso in the usual way. According
to QE-GIPAW version, only newer versions of QE are supported.
You must compile both PW and NEB.
\item Download QE-GIPAW from tarball or from SVN. QE-GIPAW can be
downloaded and built outside the Quantum-Espresso folder:\\
From tarball:
\begin{verbatim}
    tar zxfv qe-gipaw-5.0.3.tar.gz
    cd qe-gipaw-5.0.3
\end{verbatim}
From SVN:
\begin{verbatim}
    svn checkout svn://cvs.qe-forge.org/scmrepos/svn/qe-gipaw/trunk qe-gipaw
    cd qe-gipaw
\end{verbatim}
\item Configure and build QE-GIPAW:
\begin{verbatim}
    ./configure --with-qe-source=quantum espresso folder containing make.sys
    (for example: ./configure --with-qe-source=$HOME/Codes/espresso-5.0.3)
    make
\end{verbatim}
\end{enumerate}
QE-GIPAW will be built according to the options and libraries specified
in \texttt{make.sys} and the \texttt{gipaw.x} executable will be placed
in the \texttt{bin} folder.

\subsection{Configure options}
\begin{itemize}
  \item \verb+--enable-band-parallel[=yes|no]+: enable parallelization
  over electronic bands (EXPERIMENTAL). The number of band groups is
  given by the \texttt{-bgrp N} command line option of \texttt{gipaw.x}.
\end{itemize}


%===================================================================================================
\section{Quick start}
%===================================================================================================
To calculate NMR/EPR parameters you need: 
\begin{enumerate}
\item pseudopotentials containing the GIPAW reconstruction (look into
folder \texttt{pseudo})
\item run \texttt{pw.x} to perform the SCF calculation 
\item run \texttt{gipaw.x} to calculate parameters (look into folder
\texttt{examples} for NMR shielding, EFG, EPR g-tensor and hyperfine
couplings)
\end{enumerate}

%===================================================================================================
\section{Input file description}
%===================================================================================================
The input file consists on only one namelist \texttt{\&inputgipaw} with the
following keywords:
\begin{description}
\item[job] (type: character, default: \texttt{'nmr'})\\
Description: select calculation to perform. The possible values are:\\
\begin{tabular}{ll}
  \texttt{'f-sum'} & check the f-sum rule\\
  \texttt{'nmr'}   & compute the magnetic susceptibility and NMR chemical shifts\\
  \texttt{'efg'}   & compute the electric field gradients at the nuclei\\
  \texttt{'g\_tensor'}& compute the EPR g-tensor\\
  \texttt{'hyperfine'}& compute the hyperfine couplings
\end{tabular}

\item[prefix] (type: character, default: \texttt{'pwscf'})\\
Description: prefix of files saved by program \texttt{pw.x}

\item[tmp\_dir] (type: character, default: \texttt{'./scratch/'})\\
Description: temporary directory for \texttt{pw.x} restart files

\item[max\_seconds] (type: real, default: 10$^7$)\\
Description: max wall time clock before writing the checkpoint and terminate

\item[restart\_mode] (type: character, default: \texttt{'restart'})\\
Description: if \texttt{'restart'} attempt to restart froma a previous
interrupted run. If \texttt{'from\_scratch'}, discard any restart information

\item[conv\_threshold] (type: real, default: 10$^{-14}$, units: Ry$^2$)\\
Description: convergence threshold for the diagonalization and for the Green's function solver

\item[diagonalization] (type: string, default: \texttt{'david'})\\
Description: diagonalization method (allowed values: \texttt{'david'} or \texttt{'cg'})

\item[isolve] (type: integer, default: 0, OBSOLETE, use diagonalization instead)\\
Description: diagonalization method (Davidson = 0, CG = 1)

\item[q\_gipaw] (type: real, default: 0.01, units: bohrradius$^{-1}$)\\
Description: the small wave-vector for linear response

\item[verbosity] (type: string default: \texttt{'low'}i)\\
Description: verbosity level (allowed values: \texttt{'low'}, \texttt{'medium'}, \texttt{'high'})

\item[iverbosity] (type: integer, default: 0, OBSOLETE, use verbosity instead)\\
Description: if iverbosity {\textgreater} 0 print debug information in output

\item[filcurr] (type: character, default: \texttt{''})\\
Description: write the induced current in this file

\item[filfield] (type: character, default: \texttt{''})\\
Description: write the induced magnetic field in this file

\item[filnics] (type: character, default: \texttt{''})\\
Description: write the NICS (Nuclear Independent Chemical Shielding) in this file
in a format suitable for the \texttt{PP.x} code

\item[use\_nmr\_macroscopic\_shape] (type: logical, default: \texttt{.false.})\\
Description: correct the chemical shift by taking into account the
macroscopic shape of the sample

\item[nmr\_macroscopic\_shape(3,3)] (type: real, default: 2/3)\\
Description: shape tensor for the macroscopic shape correction

\item[spline\_ps] (type: logical, default: \texttt{.true.})\\
Description: interpolate pseudopotentials with cubic splines

\item[q\_efg(1..ntyp)] (type: real, default: 1.0, units: 10$^{-30}$~m$^2$ = 0.01 barn)\\
Description: for each atomic specie, the nuclear quadrupole

\item[hfi\_output\_unit] (type: character, default: \texttt{'MHz'})\\
Description: units for hyperfine couplings in output. The possible values are: 
\texttt{'MHz'}, \texttt{'mT'}, \texttt{'G'}, \texttt{'Gauss'}, \texttt{'10e-4cm-1'}

\item[hfi\_nuclear\_g\_factor(1..ntyp)] (type: real, default: 0.0)\\
Description: for each atomic specie, the nuclear g-factor

\item[core\_relax\_method] (type: integer, default: 1)\\
Description: select the method to evaluate the core polarization contribution
to the isotropic hyperfine (Fermi contact). The possible values are:\\
\begin{tabular}{ll}
  1 & perturbative~\cite{bahramy07}, exchange-only (Slater X-$\alpha$)\\
  2 & perturbative~\cite{bahramy07}, exchange-only\\
  3 & perturbative~\cite{bahramy07}, exchange and correlation\\
\end{tabular}

\end{description}
There a number of obsolete or development variables that can be removed at any
time from the code: \textbf{radial\_integral\_splines,
hfi\_via\_reconstruction\_only, hfi\_extrapolation\_npoints, pawproj(1..ntyp),
read\_recon\_in\_paratec\_fmt, file\_reconstruction(1..ntyp), isolve, iverbosity}.


%===================================================================================================
\section{Limitations}
%===================================================================================================
Symmetry operations that do not map cartesian axes are not allowed
(i.e. 120$^\circ$ rotations). If you have a triclinc cell, remove all
symmetries (\texttt{nosym = .true.}). In the special case of an hexagonal
cell, you can use \texttt{ibrav = 0} and orient the cell like in the
quartz example. The keyword \texttt{CELL\_PARAMETERS cubic} prevents PW
to detect 120$^\circ$ rotations.


%===================================================================================================
\section{Resources}
%===================================================================================================
\begin{itemize}
\item Websites: \url{http://qe-forge.org/projects/qe-gipaw}, \url{http://www.gipaw.net/}
\item NMR periodic table: \url{http://www.pascal-man.com/periodic-table/periodictable.html}
\item Tutorials: \url{http://www.gipaw.net/work_zurich09.html},
                 \url{http://sites.google.com/site/cecamspectra2010/program} (day 2),
                 \url{http://www.cecam.org/workshop-868.html}
\end{itemize}


%===================================================================================================
\begin{thebibliography}{99}
%===================================================================================================
\bibitem{pickard01}
  C. J. Pickard and F. Mauri, Phys. Rev. B \textbf{63}, 245101 (2001)

\bibitem{yates07}
  J. R. Yates, C. J. Pickard and F. Mauri, Phys. Rev. B \textbf{76}, 024401 (2007)

\bibitem{pickard02}
  C. J. Pickard and F. Mauri, Phys. Rev. Lett. \textbf{88}, 086403 (2002)

%\bibitem{davezac07}
%  M. d'Avezac, N. Marzari and F. Mauri, Phys. Rev. B \textbf{76}, 165122 (2007)

\bibitem{bahramy07}
  M. S. Bahramy, M. H. F. Sluiter and Y. Kawazoe, Phys. Rev. B  \textbf{76}, 035124 (2007)
\end{thebibliography}

%===================================================================================================
\end{document}
%===================================================================================================

